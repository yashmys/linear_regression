\documentclass[]{article}
\usepackage{lmodern}
\usepackage{amssymb,amsmath}
\usepackage{ifxetex,ifluatex}
\usepackage{fixltx2e} % provides \textsubscript
\ifnum 0\ifxetex 1\fi\ifluatex 1\fi=0 % if pdftex
  \usepackage[T1]{fontenc}
  \usepackage[utf8]{inputenc}
\else % if luatex or xelatex
  \ifxetex
    \usepackage{mathspec}
  \else
    \usepackage{fontspec}
  \fi
  \defaultfontfeatures{Ligatures=TeX,Scale=MatchLowercase}
\fi
% use upquote if available, for straight quotes in verbatim environments
\IfFileExists{upquote.sty}{\usepackage{upquote}}{}
% use microtype if available
\IfFileExists{microtype.sty}{%
\usepackage{microtype}
\UseMicrotypeSet[protrusion]{basicmath} % disable protrusion for tt fonts
}{}
\usepackage[margin=1in]{geometry}
\usepackage{hyperref}
\hypersetup{unicode=true,
            pdftitle={linear\_regression.R},
            pdfauthor={Dhathri},
            pdfborder={0 0 0},
            breaklinks=true}
\urlstyle{same}  % don't use monospace font for urls
\usepackage{color}
\usepackage{fancyvrb}
\newcommand{\VerbBar}{|}
\newcommand{\VERB}{\Verb[commandchars=\\\{\}]}
\DefineVerbatimEnvironment{Highlighting}{Verbatim}{commandchars=\\\{\}}
% Add ',fontsize=\small' for more characters per line
\usepackage{framed}
\definecolor{shadecolor}{RGB}{248,248,248}
\newenvironment{Shaded}{\begin{snugshade}}{\end{snugshade}}
\newcommand{\KeywordTok}[1]{\textcolor[rgb]{0.13,0.29,0.53}{\textbf{#1}}}
\newcommand{\DataTypeTok}[1]{\textcolor[rgb]{0.13,0.29,0.53}{#1}}
\newcommand{\DecValTok}[1]{\textcolor[rgb]{0.00,0.00,0.81}{#1}}
\newcommand{\BaseNTok}[1]{\textcolor[rgb]{0.00,0.00,0.81}{#1}}
\newcommand{\FloatTok}[1]{\textcolor[rgb]{0.00,0.00,0.81}{#1}}
\newcommand{\ConstantTok}[1]{\textcolor[rgb]{0.00,0.00,0.00}{#1}}
\newcommand{\CharTok}[1]{\textcolor[rgb]{0.31,0.60,0.02}{#1}}
\newcommand{\SpecialCharTok}[1]{\textcolor[rgb]{0.00,0.00,0.00}{#1}}
\newcommand{\StringTok}[1]{\textcolor[rgb]{0.31,0.60,0.02}{#1}}
\newcommand{\VerbatimStringTok}[1]{\textcolor[rgb]{0.31,0.60,0.02}{#1}}
\newcommand{\SpecialStringTok}[1]{\textcolor[rgb]{0.31,0.60,0.02}{#1}}
\newcommand{\ImportTok}[1]{#1}
\newcommand{\CommentTok}[1]{\textcolor[rgb]{0.56,0.35,0.01}{\textit{#1}}}
\newcommand{\DocumentationTok}[1]{\textcolor[rgb]{0.56,0.35,0.01}{\textbf{\textit{#1}}}}
\newcommand{\AnnotationTok}[1]{\textcolor[rgb]{0.56,0.35,0.01}{\textbf{\textit{#1}}}}
\newcommand{\CommentVarTok}[1]{\textcolor[rgb]{0.56,0.35,0.01}{\textbf{\textit{#1}}}}
\newcommand{\OtherTok}[1]{\textcolor[rgb]{0.56,0.35,0.01}{#1}}
\newcommand{\FunctionTok}[1]{\textcolor[rgb]{0.00,0.00,0.00}{#1}}
\newcommand{\VariableTok}[1]{\textcolor[rgb]{0.00,0.00,0.00}{#1}}
\newcommand{\ControlFlowTok}[1]{\textcolor[rgb]{0.13,0.29,0.53}{\textbf{#1}}}
\newcommand{\OperatorTok}[1]{\textcolor[rgb]{0.81,0.36,0.00}{\textbf{#1}}}
\newcommand{\BuiltInTok}[1]{#1}
\newcommand{\ExtensionTok}[1]{#1}
\newcommand{\PreprocessorTok}[1]{\textcolor[rgb]{0.56,0.35,0.01}{\textit{#1}}}
\newcommand{\AttributeTok}[1]{\textcolor[rgb]{0.77,0.63,0.00}{#1}}
\newcommand{\RegionMarkerTok}[1]{#1}
\newcommand{\InformationTok}[1]{\textcolor[rgb]{0.56,0.35,0.01}{\textbf{\textit{#1}}}}
\newcommand{\WarningTok}[1]{\textcolor[rgb]{0.56,0.35,0.01}{\textbf{\textit{#1}}}}
\newcommand{\AlertTok}[1]{\textcolor[rgb]{0.94,0.16,0.16}{#1}}
\newcommand{\ErrorTok}[1]{\textcolor[rgb]{0.64,0.00,0.00}{\textbf{#1}}}
\newcommand{\NormalTok}[1]{#1}
\usepackage{graphicx,grffile}
\makeatletter
\def\maxwidth{\ifdim\Gin@nat@width>\linewidth\linewidth\else\Gin@nat@width\fi}
\def\maxheight{\ifdim\Gin@nat@height>\textheight\textheight\else\Gin@nat@height\fi}
\makeatother
% Scale images if necessary, so that they will not overflow the page
% margins by default, and it is still possible to overwrite the defaults
% using explicit options in \includegraphics[width, height, ...]{}
\setkeys{Gin}{width=\maxwidth,height=\maxheight,keepaspectratio}
\IfFileExists{parskip.sty}{%
\usepackage{parskip}
}{% else
\setlength{\parindent}{0pt}
\setlength{\parskip}{6pt plus 2pt minus 1pt}
}
\setlength{\emergencystretch}{3em}  % prevent overfull lines
\providecommand{\tightlist}{%
  \setlength{\itemsep}{0pt}\setlength{\parskip}{0pt}}
\setcounter{secnumdepth}{0}
% Redefines (sub)paragraphs to behave more like sections
\ifx\paragraph\undefined\else
\let\oldparagraph\paragraph
\renewcommand{\paragraph}[1]{\oldparagraph{#1}\mbox{}}
\fi
\ifx\subparagraph\undefined\else
\let\oldsubparagraph\subparagraph
\renewcommand{\subparagraph}[1]{\oldsubparagraph{#1}\mbox{}}
\fi

%%% Use protect on footnotes to avoid problems with footnotes in titles
\let\rmarkdownfootnote\footnote%
\def\footnote{\protect\rmarkdownfootnote}

%%% Change title format to be more compact
\usepackage{titling}

% Create subtitle command for use in maketitle
\newcommand{\subtitle}[1]{
  \posttitle{
    \begin{center}\large#1\end{center}
    }
}

\setlength{\droptitle}{-2em}
  \title{linear\_regression.R}
  \pretitle{\vspace{\droptitle}\centering\huge}
  \posttitle{\par}
  \author{Dhathri}
  \preauthor{\centering\large\emph}
  \postauthor{\par}
  \predate{\centering\large\emph}
  \postdate{\par}
  \date{Mon May 28 18:26:34 2018}


\begin{document}
\maketitle

\begin{Shaded}
\begin{Highlighting}[]
\CommentTok{#  Introduction}
\NormalTok{## ══════════════}

\CommentTok{#   • Learning objectives:}
\NormalTok{##     • Learn the R formula interface}
\NormalTok{##     • Specify factor contrasts to test specific hypotheses}
\NormalTok{##     • Perform model comparisons}
\NormalTok{##     • Run and interpret variety of regression models in R}

\NormalTok{## Set working directory}
\NormalTok{## ─────────────────────────}

\NormalTok{##   It is often helpful to start your R session by setting your working}
\NormalTok{##   directory so you don't have to type the full path names to your data}
\NormalTok{##   and other files}

\CommentTok{# set the working directory}
\CommentTok{# setwd("~/Desktop/Rstatistics")}
\CommentTok{# setwd("C:/Users/dataclass/Desktop/Rstatistics")}

\NormalTok{##   You might also start by listing the files in your working directory}
\KeywordTok{setwd}\NormalTok{(}\StringTok{"C:}\CharTok{\textbackslash{}\textbackslash{}}\StringTok{Users}\CharTok{\textbackslash{}\textbackslash{}}\StringTok{Dhathri}\CharTok{\textbackslash{}\textbackslash{}}\StringTok{Desktop}\CharTok{\textbackslash{}\textbackslash{}}\StringTok{R}\CharTok{\textbackslash{}\textbackslash{}}\StringTok{linear_regression"}\NormalTok{)}
\KeywordTok{getwd}\NormalTok{() }\CommentTok{# where am I?}
\end{Highlighting}
\end{Shaded}

\begin{verbatim}
## [1] "C:/Users/Dhathri/Desktop/R/linear_regression"
\end{verbatim}

\begin{Shaded}
\begin{Highlighting}[]
\KeywordTok{list.files}\NormalTok{(}\StringTok{"dataSets"}\NormalTok{) }\CommentTok{# files in the dataSets folder}
\end{Highlighting}
\end{Shaded}

\begin{verbatim}
## [1] "Exam.rds"   "states.dta" "states.rds"
\end{verbatim}

\begin{Shaded}
\begin{Highlighting}[]
\NormalTok{## Load the states data}
\NormalTok{## ────────────────────────}

\CommentTok{# read the states data}
\NormalTok{states.data <-}\StringTok{ }\KeywordTok{readRDS}\NormalTok{(}\StringTok{"dataSets/states.rds"}\NormalTok{) }
\CommentTok{#get labels}
\NormalTok{states.info <-}\StringTok{ }\KeywordTok{data.frame}\NormalTok{(}\KeywordTok{attributes}\NormalTok{(states.data)[}\KeywordTok{c}\NormalTok{(}\StringTok{"names"}\NormalTok{, }\StringTok{"var.labels"}\NormalTok{)])}
\CommentTok{#look at last few labels}
\KeywordTok{tail}\NormalTok{(states.info, }\DecValTok{8}\NormalTok{)}
\end{Highlighting}
\end{Shaded}

\begin{verbatim}
##      names                      var.labels
## 14    csat        Mean composite SAT score
## 15    vsat           Mean verbal SAT score
## 16    msat             Mean math SAT score
## 17 percent       % HS graduates taking SAT
## 18 expense Per pupil expenditures prim&sec
## 19  income Median household income, $1,000
## 20    high             % adults HS diploma
## 21 college         % adults college degree
\end{verbatim}

\begin{Shaded}
\begin{Highlighting}[]
\NormalTok{## Linear regression}
\NormalTok{## ═══════════════════}

\NormalTok{## Examine the data before fitting models}
\NormalTok{## ────────,──────────────────────────────────}

\NormalTok{##   Start by examining the data to check for problems.}

\CommentTok{# summary of expense and csat columns, all rows}
\NormalTok{sts.ex.sat <-}\StringTok{ }\KeywordTok{subset}\NormalTok{(states.data, }\DataTypeTok{select =} \KeywordTok{c}\NormalTok{(}\StringTok{"expense"}\NormalTok{, }\StringTok{"csat"}\NormalTok{))}
\KeywordTok{summary}\NormalTok{(sts.ex.sat)}
\end{Highlighting}
\end{Shaded}

\begin{verbatim}
##     expense          csat       
##  Min.   :2960   Min.   : 832.0  
##  1st Qu.:4352   1st Qu.: 888.0  
##  Median :5000   Median : 926.0  
##  Mean   :5236   Mean   : 944.1  
##  3rd Qu.:5794   3rd Qu.: 997.0  
##  Max.   :9259   Max.   :1093.0
\end{verbatim}

\begin{Shaded}
\begin{Highlighting}[]
\CommentTok{# correlation between expense and csat}
\KeywordTok{cor}\NormalTok{(sts.ex.sat)}
\end{Highlighting}
\end{Shaded}

\begin{verbatim}
##            expense       csat
## expense  1.0000000 -0.4662978
## csat    -0.4662978  1.0000000
\end{verbatim}

\begin{Shaded}
\begin{Highlighting}[]
\NormalTok{## Plot the data before fitting models}
\NormalTok{## ───────────────────────────────────────}

\NormalTok{##   Plot the data to look for multivariate outliers, non-linear}
\NormalTok{##   relationships etc.}

\CommentTok{# scatter plot of expense vs csat}
\KeywordTok{plot}\NormalTok{(sts.ex.sat)}
\end{Highlighting}
\end{Shaded}

\includegraphics{linear_regression_files/figure-latex/unnamed-chunk-1-1.pdf}

\begin{Shaded}
\begin{Highlighting}[]
\NormalTok{## Linear regression example}
\NormalTok{## ─────────────────────────────}

\NormalTok{##   • Linear regression models can be fit with the `lm()' function}
\NormalTok{##   • For example, we can use `lm' to predict SAT scores based on}
\NormalTok{##     per-pupal expenditures:}

\CommentTok{# Fit our regression model}
\NormalTok{sat.mod <-}\StringTok{ }\KeywordTok{lm}\NormalTok{(csat }\OperatorTok{~}\StringTok{ }\NormalTok{expense, }\CommentTok{# regression formula}
              \DataTypeTok{data=}\NormalTok{states.data) }\CommentTok{# data set}
\CommentTok{# Summarize and print the results}
\KeywordTok{summary}\NormalTok{(sat.mod) }\CommentTok{# show regression coefficients table}
\end{Highlighting}
\end{Shaded}

\begin{verbatim}
## 
## Call:
## lm(formula = csat ~ expense, data = states.data)
## 
## Residuals:
##      Min       1Q   Median       3Q      Max 
## -131.811  -38.085    5.607   37.852  136.495 
## 
## Coefficients:
##               Estimate Std. Error t value Pr(>|t|)    
## (Intercept)  1.061e+03  3.270e+01   32.44  < 2e-16 ***
## expense     -2.228e-02  6.037e-03   -3.69 0.000563 ***
## ---
## Signif. codes:  0 '***' 0.001 '**' 0.01 '*' 0.05 '.' 0.1 ' ' 1
## 
## Residual standard error: 59.81 on 49 degrees of freedom
## Multiple R-squared:  0.2174, Adjusted R-squared:  0.2015 
## F-statistic: 13.61 on 1 and 49 DF,  p-value: 0.0005631
\end{verbatim}

\begin{Shaded}
\begin{Highlighting}[]
\NormalTok{## Why is the association between expense and SAT scores /negative/?}
\NormalTok{## ─────────────────────────────────────────────────────────────────────}

\NormalTok{##   Many people find it surprising that the per-capita expenditure on}
\NormalTok{##   students is negatively related to SAT scores. The beauty of multiple}
\NormalTok{##   regression is that we can try to pull these apart. What would the}
\NormalTok{##   association between expense and SAT scores be if there were no}
\NormalTok{##   difference among the states in the percentage of students taking the}
\NormalTok{##   SAT?}

\KeywordTok{summary}\NormalTok{(}\KeywordTok{lm}\NormalTok{(csat }\OperatorTok{~}\StringTok{ }\NormalTok{expense }\OperatorTok{+}\StringTok{ }\NormalTok{percent, }\DataTypeTok{data =}\NormalTok{ states.data))}
\end{Highlighting}
\end{Shaded}

\begin{verbatim}
## 
## Call:
## lm(formula = csat ~ expense + percent, data = states.data)
## 
## Residuals:
##     Min      1Q  Median      3Q     Max 
## -62.921 -24.318   1.741  15.502  75.623 
## 
## Coefficients:
##               Estimate Std. Error t value Pr(>|t|)    
## (Intercept) 989.807403  18.395770  53.806  < 2e-16 ***
## expense       0.008604   0.004204   2.046   0.0462 *  
## percent      -2.537700   0.224912 -11.283 4.21e-15 ***
## ---
## Signif. codes:  0 '***' 0.001 '**' 0.01 '*' 0.05 '.' 0.1 ' ' 1
## 
## Residual standard error: 31.62 on 48 degrees of freedom
## Multiple R-squared:  0.7857, Adjusted R-squared:  0.7768 
## F-statistic: 88.01 on 2 and 48 DF,  p-value: < 2.2e-16
\end{verbatim}

\begin{Shaded}
\begin{Highlighting}[]
\NormalTok{## The lm class and methods}
\NormalTok{## ────────────────────────────}

\NormalTok{##   OK, we fit our model. Now what?}
\NormalTok{##   • Examine the model object:}

\KeywordTok{class}\NormalTok{(sat.mod)}
\end{Highlighting}
\end{Shaded}

\begin{verbatim}
## [1] "lm"
\end{verbatim}

\begin{Shaded}
\begin{Highlighting}[]
\KeywordTok{names}\NormalTok{(sat.mod)}
\end{Highlighting}
\end{Shaded}

\begin{verbatim}
##  [1] "coefficients"  "residuals"     "effects"       "rank"         
##  [5] "fitted.values" "assign"        "qr"            "df.residual"  
##  [9] "xlevels"       "call"          "terms"         "model"
\end{verbatim}

\begin{Shaded}
\begin{Highlighting}[]
\KeywordTok{methods}\NormalTok{(}\DataTypeTok{class =} \KeywordTok{class}\NormalTok{(sat.mod))[}\DecValTok{1}\OperatorTok{:}\DecValTok{9}\NormalTok{]}
\end{Highlighting}
\end{Shaded}

\begin{verbatim}
## [1] "add1.lm"                   "alias.lm"                 
## [3] "anova.lm"                  "case.names.lm"            
## [5] "coerce,oldClass,S3-method" "confint.lm"               
## [7] "cooks.distance.lm"         "deviance.lm"              
## [9] "dfbeta.lm"
\end{verbatim}

\begin{Shaded}
\begin{Highlighting}[]
\NormalTok{##   • Use function methods to get more information about the fit}

\KeywordTok{confint}\NormalTok{(sat.mod)}
\end{Highlighting}
\end{Shaded}

\begin{verbatim}
##                    2.5 %        97.5 %
## (Intercept) 995.01753164 1126.44735626
## expense      -0.03440768   -0.01014361
\end{verbatim}

\begin{Shaded}
\begin{Highlighting}[]
\CommentTok{# hist(residuals(sat.mod))}

\NormalTok{## Linear Regression Assumptions}
\NormalTok{## ─────────────────────────────────}

\NormalTok{##   • Ordinary least squares regression relies on several assumptions,}
\NormalTok{##     including that the residuals are normally distributed and}
\NormalTok{##     homoscedastic, the errors are independent and the relationships are}
\NormalTok{##     linear.}

\NormalTok{##   • Investigate these assumptions visually by plotting your model:}

\KeywordTok{par}\NormalTok{(}\DataTypeTok{mar =} \KeywordTok{c}\NormalTok{(}\DecValTok{4}\NormalTok{, }\DecValTok{4}\NormalTok{, }\DecValTok{2}\NormalTok{, }\DecValTok{2}\NormalTok{), }\DataTypeTok{mfrow =} \KeywordTok{c}\NormalTok{(}\DecValTok{1}\NormalTok{, }\DecValTok{2}\NormalTok{)) }\CommentTok{#optional}
\KeywordTok{plot}\NormalTok{(sat.mod, }\DataTypeTok{which =} \KeywordTok{c}\NormalTok{(}\DecValTok{1}\NormalTok{, }\DecValTok{2}\NormalTok{)) }\CommentTok{# "which" argument optional}
\end{Highlighting}
\end{Shaded}

\includegraphics{linear_regression_files/figure-latex/unnamed-chunk-1-2.pdf}

\begin{Shaded}
\begin{Highlighting}[]
\NormalTok{## Comparing models}
\NormalTok{## ────────────────────}

\NormalTok{##   Do congressional voting patterns predict SAT scores over and above}
\NormalTok{##   expense? Fit two models and compare them:}

\CommentTok{# fit another model, adding house and senate as predictors}
\NormalTok{sat.voting.mod <-}\StringTok{  }\KeywordTok{lm}\NormalTok{(csat }\OperatorTok{~}\StringTok{ }\NormalTok{expense }\OperatorTok{+}\StringTok{ }\NormalTok{house }\OperatorTok{+}\StringTok{ }\NormalTok{senate,}
                      \DataTypeTok{data =} \KeywordTok{na.omit}\NormalTok{(states.data))}
\NormalTok{sat.mod <-}\StringTok{ }\KeywordTok{update}\NormalTok{(sat.mod, }\DataTypeTok{data=}\KeywordTok{na.omit}\NormalTok{(states.data))}
\CommentTok{# compare using the anova() function}
\KeywordTok{anova}\NormalTok{(sat.mod, sat.voting.mod)}
\end{Highlighting}
\end{Shaded}

\begin{verbatim}
## Analysis of Variance Table
## 
## Model 1: csat ~ expense
## Model 2: csat ~ expense + house + senate
##   Res.Df    RSS Df Sum of Sq      F  Pr(>F)  
## 1     46 169050                              
## 2     44 149284  2     19766 2.9128 0.06486 .
## ---
## Signif. codes:  0 '***' 0.001 '**' 0.01 '*' 0.05 '.' 0.1 ' ' 1
\end{verbatim}

\begin{Shaded}
\begin{Highlighting}[]
\KeywordTok{coef}\NormalTok{(}\KeywordTok{summary}\NormalTok{(sat.voting.mod))}
\end{Highlighting}
\end{Shaded}

\begin{verbatim}
##                  Estimate   Std. Error    t value     Pr(>|t|)
## (Intercept) 1082.93438041 38.633812740 28.0307405 1.067795e-29
## expense       -0.01870832  0.009691494 -1.9303852 6.001998e-02
## house         -1.44243754  0.600478382 -2.4021473 2.058666e-02
## senate         0.49817861  0.513561356  0.9700469 3.373256e-01
\end{verbatim}

\begin{Shaded}
\begin{Highlighting}[]
\NormalTok{## Exercise: least squares regression}
\NormalTok{## ────────────────────────────────────────}

\NormalTok{##   Use the /states.rds/ data set. Fit a model predicting energy consumed}
\NormalTok{##   per capita (energy) from the percentage of residents living in}
\NormalTok{##   metropolitan areas (metro). Be sure to}
\NormalTok{##   1. Examine/plot the data before fitting the model}
\NormalTok{##   2. Print and interpret the model `summary'}
\NormalTok{##   3. `plot' the model to look for deviations from modeling assumptions}
\NormalTok{sts.egy.pc <-}\StringTok{  }\KeywordTok{subset}\NormalTok{(states.data , }\DataTypeTok{select =} \KeywordTok{c}\NormalTok{(}\StringTok{"energy"}\NormalTok{, }\StringTok{"metro"}\NormalTok{))}
\KeywordTok{summary}\NormalTok{(sts.egy.pc)}
\end{Highlighting}
\end{Shaded}

\begin{verbatim}
##      energy          metro       
##  Min.   :200.0   Min.   : 20.40  
##  1st Qu.:285.0   1st Qu.: 46.98  
##  Median :320.0   Median : 67.55  
##  Mean   :354.5   Mean   : 64.07  
##  3rd Qu.:371.5   3rd Qu.: 81.58  
##  Max.   :991.0   Max.   :100.00  
##  NA's   :1       NA's   :1
\end{verbatim}

\begin{Shaded}
\begin{Highlighting}[]
\KeywordTok{plot}\NormalTok{(sts.egy.pc)}

\NormalTok{egy.mod <-}\StringTok{ }\KeywordTok{lm}\NormalTok{(energy }\OperatorTok{~}\StringTok{ }\NormalTok{metro , }\DataTypeTok{data =} \KeywordTok{na.omit}\NormalTok{(states.data))}
\KeywordTok{summary}\NormalTok{(egy.mod)}
\end{Highlighting}
\end{Shaded}

\begin{verbatim}
## 
## Call:
## lm(formula = energy ~ metro, data = na.omit(states.data))
## 
## Residuals:
##     Min      1Q  Median      3Q     Max 
## -179.17  -54.21  -21.64   15.07  448.02 
## 
## Coefficients:
##             Estimate Std. Error t value Pr(>|t|)    
## (Intercept) 449.8382    50.4472   8.917 1.37e-11 ***
## metro        -1.6526     0.7428  -2.225    0.031 *  
## ---
## Signif. codes:  0 '***' 0.001 '**' 0.01 '*' 0.05 '.' 0.1 ' ' 1
## 
## Residual standard error: 112.3 on 46 degrees of freedom
## Multiple R-squared:  0.09714,    Adjusted R-squared:  0.07751 
## F-statistic: 4.949 on 1 and 46 DF,  p-value: 0.03105
\end{verbatim}

\begin{Shaded}
\begin{Highlighting}[]
\KeywordTok{plot}\NormalTok{(egy.mod)}
\end{Highlighting}
\end{Shaded}

\includegraphics{linear_regression_files/figure-latex/unnamed-chunk-1-3.pdf}
\includegraphics{linear_regression_files/figure-latex/unnamed-chunk-1-4.pdf}

\begin{Shaded}
\begin{Highlighting}[]
\NormalTok{##   Select one or more additional predictors to add to your model and}
\NormalTok{##   repeat steps 1-3. Is this model significantly better than the model}
\NormalTok{##   with /metro/ as the only predictor?anova(egy.mod,emod2)}
\NormalTok{emod2  <-}\StringTok{  }\KeywordTok{lm}\NormalTok{(energy }\OperatorTok{~}\StringTok{ }\NormalTok{metro }\OperatorTok{+}\StringTok{  }\NormalTok{green }\OperatorTok{+}\StringTok{ }\NormalTok{toxic }\OperatorTok{+}\StringTok{ }\NormalTok{area , }\DataTypeTok{data =} \KeywordTok{na.omit}\NormalTok{(states.data) )}
\KeywordTok{summary}\NormalTok{(emod2)}
\end{Highlighting}
\end{Shaded}

\begin{verbatim}
## 
## Call:
## lm(formula = energy ~ metro + green + toxic + area, data = na.omit(states.data))
## 
## Residuals:
##     Min      1Q  Median      3Q     Max 
## -180.56  -25.95   -1.20   22.87  163.31 
## 
## Coefficients:
##              Estimate Std. Error t value Pr(>|t|)    
## (Intercept) 1.474e+02  3.721e+01   3.962 0.000275 ***
## metro       2.154e-01  4.165e-01   0.517 0.607643    
## green       4.637e+00  5.737e-01   8.082 3.63e-10 ***
## toxic       2.575e+00  4.636e-01   5.554 1.62e-06 ***
## area        3.349e-04  1.824e-04   1.836 0.073262 .  
## ---
## Signif. codes:  0 '***' 0.001 '**' 0.01 '*' 0.05 '.' 0.1 ' ' 1
## 
## Residual standard error: 57.15 on 43 degrees of freedom
## Multiple R-squared:  0.7815, Adjusted R-squared:  0.7612 
## F-statistic: 38.45 on 4 and 43 DF,  p-value: 1.116e-13
\end{verbatim}

\begin{Shaded}
\begin{Highlighting}[]
\KeywordTok{plot}\NormalTok{(emod2)}
\end{Highlighting}
\end{Shaded}

\includegraphics{linear_regression_files/figure-latex/unnamed-chunk-1-5.pdf}
\includegraphics{linear_regression_files/figure-latex/unnamed-chunk-1-6.pdf}

\begin{Shaded}
\begin{Highlighting}[]
\NormalTok{## Interactions and factors}
\NormalTok{## ══════════════════════════}

\NormalTok{## Modeling interactions}
\NormalTok{## ─────────────────────────}

\NormalTok{##   Interactions allow us assess the extent to which the association}
\NormalTok{##   between one predictor and the outcome depends on a second predictor.}
\NormalTok{##   For example: Does the association between expense and SAT scores}
\NormalTok{##   depend on the median income in the state?}

  \CommentTok{#Add the interaction to the model}
\NormalTok{sat.expense.by.percent <-}\StringTok{ }\KeywordTok{lm}\NormalTok{(csat }\OperatorTok{~}\StringTok{ }\NormalTok{expense}\OperatorTok{*}\NormalTok{income,}
                             \DataTypeTok{data=}\NormalTok{states.data) }
\CommentTok{#Show the results}
  \KeywordTok{coef}\NormalTok{(}\KeywordTok{summary}\NormalTok{(sat.expense.by.percent)) }\CommentTok{# show regression coefficients table}
\end{Highlighting}
\end{Shaded}

\begin{verbatim}
##                     Estimate   Std. Error   t value     Pr(>|t|)
## (Intercept)     1.380364e+03 1.720863e+02  8.021351 2.367069e-10
## expense        -6.384067e-02 3.270087e-02 -1.952262 5.687837e-02
## income         -1.049785e+01 4.991463e+00 -2.103161 4.083253e-02
## expense:income  1.384647e-03 8.635529e-04  1.603431 1.155395e-01
\end{verbatim}

\begin{Shaded}
\begin{Highlighting}[]
\NormalTok{## Regression with categorical predictors}
\NormalTok{## ──────────────────────────────────────────}

\NormalTok{##   Let's try to predict SAT scores from region, a categorical variable.}
\NormalTok{##   Note that you must make sure R does not think your categorical}
\NormalTok{##   variable is numeric.}

\CommentTok{# make sure R knows region is categorical}
\KeywordTok{str}\NormalTok{(states.data}\OperatorTok{$}\NormalTok{region)}
\end{Highlighting}
\end{Shaded}

\begin{verbatim}
##  Factor w/ 4 levels "West","N. East",..: 3 1 1 3 1 1 2 3 NA 3 ...
\end{verbatim}

\begin{Shaded}
\begin{Highlighting}[]
\NormalTok{states.data}\OperatorTok{$}\NormalTok{region <-}\StringTok{ }\KeywordTok{factor}\NormalTok{(states.data}\OperatorTok{$}\NormalTok{region)}
\CommentTok{#Add region to the model}
\NormalTok{sat.region <-}\StringTok{ }\KeywordTok{lm}\NormalTok{(csat }\OperatorTok{~}\StringTok{ }\NormalTok{region,}
                 \DataTypeTok{data=}\NormalTok{states.data) }
\CommentTok{#Show the results}
\KeywordTok{coef}\NormalTok{(}\KeywordTok{summary}\NormalTok{(sat.region)) }\CommentTok{# show regression coefficients table}
\end{Highlighting}
\end{Shaded}

\begin{verbatim}
##                Estimate Std. Error    t value     Pr(>|t|)
## (Intercept)   946.30769   14.79582 63.9577807 1.352577e-46
## regionN. East -56.75214   23.13285 -2.4533141 1.800383e-02
## regionSouth   -16.30769   19.91948 -0.8186806 4.171898e-01
## regionMidwest  63.77564   21.35592  2.9863209 4.514152e-03
\end{verbatim}

\begin{Shaded}
\begin{Highlighting}[]
\KeywordTok{anova}\NormalTok{(sat.region) }\CommentTok{# show ANOVA table}
\end{Highlighting}
\end{Shaded}

\begin{verbatim}
## Analysis of Variance Table
## 
## Response: csat
##           Df Sum Sq Mean Sq F value    Pr(>F)    
## region     3  82049 27349.8  9.6102 4.859e-05 ***
## Residuals 46 130912  2845.9                      
## ---
## Signif. codes:  0 '***' 0.001 '**' 0.01 '*' 0.05 '.' 0.1 ' ' 1
\end{verbatim}

\begin{Shaded}
\begin{Highlighting}[]
\NormalTok{##   Again, *make sure to tell R which variables are categorical by}
\NormalTok{##   converting them to factors!*}

\NormalTok{## Setting factor reference groups and contrasts}
\NormalTok{## ─────────────────────────────────────────────────}

\NormalTok{##   In the previous example we use the default contrasts for region. The}
\NormalTok{##   default in R is treatment contrasts, with the first level as the}
\NormalTok{##   reference. We can change the reference group or use another coding}
\NormalTok{##   scheme using the `C' function.}

\CommentTok{# print default contrasts}
\KeywordTok{contrasts}\NormalTok{(states.data}\OperatorTok{$}\NormalTok{region)}
\end{Highlighting}
\end{Shaded}

\begin{verbatim}
##         N. East South Midwest
## West          0     0       0
## N. East       1     0       0
## South         0     1       0
## Midwest       0     0       1
\end{verbatim}

\begin{Shaded}
\begin{Highlighting}[]
\CommentTok{# change the reference group}
\KeywordTok{coef}\NormalTok{(}\KeywordTok{summary}\NormalTok{(}\KeywordTok{lm}\NormalTok{(csat }\OperatorTok{~}\StringTok{ }\KeywordTok{C}\NormalTok{(region, }\DataTypeTok{base=}\DecValTok{4}\NormalTok{),}
                \DataTypeTok{data=}\NormalTok{states.data)))}
\end{Highlighting}
\end{Shaded}

\begin{verbatim}
##                        Estimate Std. Error   t value     Pr(>|t|)
## (Intercept)          1010.08333   15.39998 65.589930 4.296307e-47
## C(region, base = 4)1  -63.77564   21.35592 -2.986321 4.514152e-03
## C(region, base = 4)2 -120.52778   23.52385 -5.123641 5.798399e-06
## C(region, base = 4)3  -80.08333   20.37225 -3.931000 2.826007e-04
\end{verbatim}

\begin{Shaded}
\begin{Highlighting}[]
\CommentTok{# change the coding scheme}
\KeywordTok{coef}\NormalTok{(}\KeywordTok{summary}\NormalTok{(}\KeywordTok{lm}\NormalTok{(csat }\OperatorTok{~}\StringTok{ }\KeywordTok{C}\NormalTok{(region, contr.helmert),}
                \DataTypeTok{data=}\NormalTok{states.data)))}
\end{Highlighting}
\end{Shaded}

\begin{verbatim}
##                             Estimate Std. Error     t value     Pr(>|t|)
## (Intercept)               943.986645   7.706155 122.4977451 1.689670e-59
## C(region, contr.helmert)1 -28.376068  11.566423  -2.4533141 1.800383e-02
## C(region, contr.helmert)2   4.022792   5.884552   0.6836191 4.976450e-01
## C(region, contr.helmert)3  22.032229   4.446777   4.9546509 1.023364e-05
\end{verbatim}

\begin{Shaded}
\begin{Highlighting}[]
\NormalTok{##   See also `?contrasts', `?contr.treatment', and `?relevel'.}

\NormalTok{## Exercise: interactions and factors}
\NormalTok{## ────────────────────────────────────────}

\NormalTok{##   Use the states data set.}

\NormalTok{##   1. Add on to the regression equation that you created in exercise 1 by}
\NormalTok{##      generating an interaction term and testing the interaction.}

\NormalTok{##  generate the interaction between metro and area }
\NormalTok{## there is no  improvement }
\NormalTok{emod3  <-}\StringTok{  }\KeywordTok{lm}\NormalTok{(energy }\OperatorTok{~}\StringTok{ }\NormalTok{metro }\OperatorTok{*}\StringTok{ }\NormalTok{area }\OperatorTok{+}\StringTok{  }\NormalTok{green }\OperatorTok{+}\StringTok{ }\NormalTok{toxic , }\DataTypeTok{data =} \KeywordTok{na.omit}\NormalTok{(states.data)  )}

\NormalTok{##   2. Try adding region to the model. Are there significant differences}
\NormalTok{##      across the four regions?}

\KeywordTok{str}\NormalTok{(states.data}\OperatorTok{$}\NormalTok{region)}
\end{Highlighting}
\end{Shaded}

\begin{verbatim}
##  Factor w/ 4 levels "West","N. East",..: 3 1 1 3 1 1 2 3 NA 3 ...
\end{verbatim}

\begin{Shaded}
\begin{Highlighting}[]
\NormalTok{states.data}\OperatorTok{$}\NormalTok{region  <-}\StringTok{  }\KeywordTok{factor}\NormalTok{(states.data}\OperatorTok{$}\NormalTok{region)}
\NormalTok{emod4   <-}\StringTok{  }\KeywordTok{lm}\NormalTok{(energy }\OperatorTok{~}\StringTok{ }\NormalTok{metro  }\OperatorTok{+}\StringTok{  }\NormalTok{green }\OperatorTok{+}\StringTok{ }\NormalTok{toxic }\OperatorTok{+}\StringTok{ }\NormalTok{region , }\DataTypeTok{data =} \KeywordTok{na.omit}\NormalTok{(states.data) )}
\KeywordTok{plot}\NormalTok{(emod4)}
\end{Highlighting}
\end{Shaded}

\includegraphics{linear_regression_files/figure-latex/unnamed-chunk-1-7.pdf}
\includegraphics{linear_regression_files/figure-latex/unnamed-chunk-1-8.pdf}

\begin{Shaded}
\begin{Highlighting}[]
\NormalTok{##  there is no impact of adding the region }
\end{Highlighting}
\end{Shaded}

\includegraphics{linear_regression_files/figure-latex/unnamed-chunk-1-9.pdf}


\end{document}
